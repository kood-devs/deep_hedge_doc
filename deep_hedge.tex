\documentclass[autodetect-engine, dvipdfmx-if-dvi, ja=standard]{ltjsarticle}

\usepackage{cite}
\usepackage{color}

\title{不完備市場におけるDeep Hedgingの応用}
\date{2020年10月6日}

\begin{document}
\maketitle

\section{はじめに}
近年、機械学習の分野における深層学習の有効性が注目されており、
ファイナンス分野における深層学習の応用にも同様に注目が集まっている。
ファイナンス分野における深層学習の主要な適用例の一つに、金融派生商品の最適ヘッジ戦略が挙げられる。
同分野については多くの先行研究があるが(先行研究の体系については、例えば\cite{ruf2020neural}を参照)、
本稿では、金融派生商品の最適ヘッジ戦略に深層学習を適用した研究一般を概観した後に、
より現実的な設定として、不完備市場における金融派生商品の最適ヘッジ戦略を扱った研究を詳細にサーベイする。


\section{深層学習を用いた最適ヘッジ戦略}
深層学習を用いた最適ヘッジ戦略については、古くから多くの研究がなされてきた。
\cite{malliaris1993neural}や\cite{hutchinson1994nonparametric}は、
ノン・パラメトリックなヘッジ手法としてニューラルネットワークに依拠したヘッジ手法を提示した嚆矢的研究である。
これらの研究では、ニューラルネットワークを用いたBlack-Scholesオプション価格の推定手法が提示されており、
ニューラルネットワークを用いることで、原資産価格過程に依らずにオプション価格の推定が可能であることが
実証されている。

最近の研究として、\cite{buehler2019deepA,buehler2019deepB}は、トレーダーが一般に用いる
情報(価格、売買シグナル、ニュース、過去の投資情報等)を特徴量に持つニューラルネットワークとして、
オプションのヘッジ戦略を定式化した。
同研究は、モデルの出力を価格ではなく取引量にしており、
損失関数に凸なリスク測度(convex risk measure)を使用している点や
摩擦のある市場に対する具体的な適用方法を明示している点(\cite{buehler2019deepB}にて詳述)が特徴である。

\cite{kolm2019dynamic}や\cite{du2020deep}も同様に、取引コストが存在する場合におけるヘッジ戦略について、
より一般的な設定のもとで検討を行っている。

\cite{ruf2020hedging}も同様に深層学習を用いたヘッジ戦略を提示している。
本研究は、DeltaとVegaの両方に着目してヘッジ戦略を構築しているほか、
Data leakageを回避する観点での学習データの扱いについても取り上げて議論している点に特徴がある。


\section{主題と直接関係ないものの面白そうな文献のメモ}
\subsection{深層学習とGAN}

\cite{wiese2019deep}は、GANによる金融時系列データ生成と、それを用いたヘッジ戦略
学習・評価について研究を行っている。

\subsection{証券指数の複製(index tracking)}
証券指数の複製(index tracking)は、金融実務において非常に関心の高い分野の一つである。
深層学習ではないものの、\cite{gendreau2019cvar}は、

\bibliography{reference.bib} 
\bibliographystyle{jplain}

\end{document}